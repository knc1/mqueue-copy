%% LyX 2.1.4 created this file.  For more info, see http://www.lyx.org/.
%% Do not edit unless you really know what you are doing.
\documentclass[english]{paper}
\usepackage{helvet}
\renewcommand{\familydefault}{\sfdefault}
\usepackage[T1]{fontenc}
\usepackage[utf8]{luainputenc}
\usepackage[letterpaper]{geometry}
\geometry{verbose,tmargin=1in,bmargin=1in,lmargin=1in,rmargin=1in}
\pagestyle{plain}
\setlength{\parskip}{\smallskipamount}
\setlength{\parindent}{0pt}
\usepackage{babel}
\usepackage[unicode=true,pdfusetitle,
 bookmarks=true,bookmarksnumbered=true,bookmarksopen=false,
 breaklinks=false,pdfborder={0 0 0},backref=false,colorlinks=false]
 {hyperref}

\makeatletter
%%%%%%%%%%%%%%%%%%%%%%%%%%%%%% Textclass specific LaTeX commands.
\newenvironment{lyxlist}[1]
{\begin{list}{}
{\settowidth{\labelwidth}{#1}
 \setlength{\leftmargin}{\labelwidth}
 \addtolength{\leftmargin}{\labelsep}
 \renewcommand{\makelabel}[1]{##1\hfil}}}
{\end{list}}

\@ifundefined{date}{}{\date{}}
\AtBeginDocument{
  \def\labelitemii{\(\circ\)}
}

\makeatother

\begin{document}
\tableofcontents{}


\section{Name}

\emph{mq\_copy} - Command line interface to Linux, POSIX, message
queues.


\section{Description}

Provides a command line interface, usable by scripting languages,
to the Linux system of POSIX message queues.

The Linux system provides the POSIX message queues as a modified file
descriptor. The default location is under the directory: \emph{/dev/mqueue}
.

POSIX message queues have kernel persistence. If not removed by \emph{unlink},
the queue will exist until the system is shut down. The action: \emph{``unlink
on close''} is a \emph{mq\_copy} command line option.

POSIX message queues allow processes to exchange data in the form
of messages.

\emph{mq\_copy} supports line oriented, string messages. That is,
lines terminated by the new-line character.

\emph{mq\_copy} provides two basic functions: \emph{copy\_in} and
\emph{copy\_out}.
\begin{itemize}
\item The \emph{copy-in} function copies line oriented strings on \emph{stdin}
as messages to a specified queue.
\item The \emph{copy-out} function copies messages from the specified queue
as line oriented strings to \emph{stdout}.
\end{itemize}

\section{Arguments}

\emph{mq\_copy} {[}--in | --out{]} {[}options{]} \emph{/queue\_name}

The minimum command arguments are the function, either \emph{--in}
(copy\_in) or \emph{--out} (copy\_out) and the name of an existing
queue. The GLIBC library requires that the queue name begin with the
path separator character '/'.


\subsection{Function}
\begin{lyxlist}{00.00.0000}
\item [{--ci~--in~-i}] copy stdin to queue
\item [{--co~--out~-o}] copy queue to stdout
\end{lyxlist}

\subsection{Options}
\begin{description}
\item [{--create~--cr~-c}] Create queue if not already present on open
\item [{--exclusive~--ex~-x}] Create queue on open, queue must not exist.
\item [{default}] Open existing queue
\item [{--ro~-r}] Open queue as read-only
\item [{--wo~-w}] Open queue as write-only
\item [{default}] Open queue as read-write\end{description}

\end{document}
