%% LyX 2.1.4 created this file.  For more info, see http://www.lyx.org/.
%% Do not edit unless you really know what you are doing.
\documentclass[english]{paper}
\usepackage{helvet}
\renewcommand{\familydefault}{\sfdefault}
\usepackage[T1]{fontenc}
\usepackage[latin9]{inputenc}
\usepackage[letterpaper]{geometry}
\geometry{verbose,tmargin=1in,bmargin=1in,lmargin=1in,rmargin=1in}
\pagestyle{plain}
\setlength{\parskip}{\smallskipamount}
\setlength{\parindent}{0pt}
\usepackage{babel}
\usepackage{amsmath}
\usepackage[unicode=true,pdfusetitle,
 bookmarks=true,bookmarksnumbered=true,bookmarksopen=false,
 breaklinks=false,pdfborder={0 0 0},backref=false,colorlinks=false]
 {hyperref}

\makeatletter
\@ifundefined{date}{}{\date{}}
\AtBeginDocument{
  \def\labelitemii{\(\circ\)}
}

\makeatother

\begin{document}
\tableofcontents{}


\section{Name}

\emph{mq\_copy} - Command line interface to Linux, POSIX, message
queues.


\section{Description}

Provides a command line interface, usable by scripting languages,
to the Linux system of POSIX message queues.

The Linux system provides the POSIX message queues as a modified file
descriptor. The default location is under the directory: \emph{/dev/mqueue}
.

POSIX message queues have kernel persistence. If not removed by \emph{unlink},
the queue will exist until the system is shut down. The action: \emph{``unlink
on close''} is a \emph{mq\_copy} command line option.

POSIX message queues allow processes to exchange data in the form
of messages.

\emph{mq\_copy} supports line oriented, string messages. That is,
lines terminated by the new-line character.

\emph{mq\_copy} provides two basic functions: \emph{copy\_in} and
\emph{copy\_out}.
\begin{itemize}
\item The \emph{copy-in} function copies line oriented strings on \emph{stdin}
as messages to a specified queue.
\item The \emph{copy-out} function copies messages from the specified queue
as line oriented strings to \emph{stdout}.
\end{itemize}

\section{Arguments}

\qquad{}\emph{mq\_copy}\qquad{}{[} \nobreakdash-\nobreakdash-in
| \nobreakdash-\nobreakdash-out {]}\qquad{}{[} options {]}\qquad{}\emph{/queue\_name}

The minimum command arguments are the function, either \emph{--in}
(copy\_in) or \emph{--out} (copy\_out) and the name of an existing
queue. The GLIBC library requires that the queue name begin with the
path separator character '/'.


\subsection{Function}

\qquad{}{[} \nobreakdash-\nobreakdash-ci | \nobreakdash-\nobreakdash-in
| \nobreakdash-i {]}\qquad{}copy stdin to queue

\qquad{}{[} \nobreakdash-\nobreakdash-co | \nobreakdash-\nobreakdash-out
| \nobreakdash-o {]}\qquad{}copy queue to stdout


\subsection{Options}

\qquad{}{[} \nobreakdash-\nobreakdash-create | \nobreakdash-\nobreakdash-cr
| \nobreakdash-c {]}\qquad{}Create queue if not already present
on open

\qquad{}{[} \nobreakdash-\nobreakdash-exclusive | \nobreakdash-\nobreakdash-ex
| \nobreakdash-x {]} \qquad{}Create queue on open, queue must not
exist.

\qquad{}default\qquad{}Open existing queue

\qquad{}{[} \nobreakdash-\nobreakdash-ro | \nobreakdash-r {]}\qquad{}Open
queue as read-only

\qquad{}{[} \nobreakdash-\nobreakdash-wo | \nobreakdash-w {]}\qquad{}Open
queue as write-only

\qquad{}default\qquad{}Open queue as read-write
\end{document}
